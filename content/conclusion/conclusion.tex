\chapter{Conclusion and Outlook}


Because the thesis focused on the use of \gls{mfa} solutions on the web, but at the same time delimitated concepts as OpenID Connect, OAuth 2.0 and \gls{sso}, further research can be done in this topics. For instance, \gls{sso} is an important factor in enterprises, often used in conjunction with services and protocols such as \gls{ldap} or \gls{ad} from Microsoft. It has yet to be researched, whether the \wa{} is combinable with \gls{sso} providers.

In the consumer section OAuth plays a relevant aspect as, e.g., Google, Facebook and Microsoft allow the registration on a website with the account a user has on their site registered. Most recently, Apple announced support for OAuth with the Apple ID, too. A matter to clarify and study in this regard is the competition between the \wa{} and OAuth. In particular the privacy aspect is interesting, because the providing party has the knowledge on which websites the user registered with their account. This allows an extensive user tracking.\footcites[See][18]{fido-ct-2}[See][4]{osti_1257179}

One the one hand, emerging technologies such as \gls{iot} and Industry 4.0 are becoming more and more connected, but on the other hand increase the attack surface drastically. An open question for continued research is if the \wa{} is a suitable use case for strong and secure authentication which is done without user interaction, but rather by machines and computer autonomously. Further it has to be evaluated if these, especially low-power computing, devices are capable of implementing both secure channels, such as \gls{tls}, and calculations on \glspl{ec}.


All in all, this thesis showed that the internet use is at risk, because 












\begin{enumerate}
	\item not yet ready for the world (iOS, IE, only USB)
	\item two keys recommended - expensive
	\item user needs to be educated!
	\item mfa is safe most of the time - sms and email and http not!
	\item new additions in the recent months, this shows the actuality and importance of the topic, but also that it might be suitable to finally kill the passwords
\end{enumerate}