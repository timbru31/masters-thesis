\chapter{Conclusion and Outlook}

This thesis starts with a citation that highlighted that passwords are not suitable for secure authentication on the web anymore. To prove this statement, the reader has been introduced into the fundamental methods and concepts of authentication. The knowledge-based method of authentication in the form of passwords is the most common method of authentication on the internet. Further, but less used methods of authentication are, for example, possession and biometrics, but the time or location can also be used to authenticate a user. Additionally, the building blocks for the \gls{uaf}, the attestation and challenge-response authentication have been explained, followed by the first example of passwordless authentication.

Based thereupon, each of the authentication methods has been analyzed in regard to their security. \autoref{chap:one-factor-security} showed that knowledge-based authentication exposes significant threats. Since the human brain has difficulties remembering the different passwords, the user re-uses them, chooses simple passwords and variants, or stores them somewhere in plain text. Moreover, service providers hat only hash passwords without salting or peppering. It increases the chances of a successful brute-force or dictionary attack when their database is breached. However, even physical possession is not safe against theft, loss, copies, or damages and therefore cannot be used as a replacement for knowledge-based authentication. Additionally, biometrics is more expensive to use and do not protect against an impersonation, either. Further, the initialization and transmission of authentication are subject to \gls{mitm} attacks, albeit a secure communications channel is used. This reinforces these that additional security measures need to be employed.

As a countermeasure to protect a user's account even in case of leaked, hacked, or reversed credentials from a hash, different \gls{mfa} solutions have been introduced. Instead of relying on one factor, a user has to provide two or more distinct factors authentication. An example of an authentication method besides passwords is the \gls{otp}. It is an (alpha)numerical password that is only valid once. In order to generate the next password, either based on time or an event, both parties need to possess a shared secret, or the \gls{rp} calculates the \gls{otp} and sends it to the client. However, both the transmission of the \gls{otp} and the generation contain potential weaknesses. The transmission via \gls{sms} or unencrypted e-mails can be eavesdropped or phished. The \gls{ss7} network is not secure enough for the transmission of confidential information. Further, relying on third party providers for the transmission increases the attack surface for malicious insiders, social engineering, or \glspl{vcfa}. Nonetheless, the generation of the \gls{otp} on the client device can be intercepted by malware, too.

Another concept to achieve \gls{mfa} is the use of the \gls{u2f} protocol. In contrary to the \gls{otp}, this authentication relies on the public-key cryptography that is handled by dedicated security tokens. They provide resistance against phishing since the origin, the client (and token) is communicating with, is verified by the web browser, therefore taking this task off the user's duty. While the security in comparison with \glspl{otp} is increased, the usability is worse. Requiring the user to purchase a dedicated piece of hardware, as well as the difference of ports a computer and mobile phone has, and transport protocols hinder the interoperability and ease of use. Further, only a few web browsers support \gls{u2f}. The outlined security threats the \wa{} faces are all of a theoretical nature, as no known and successful exploit of vulnerabilities exists. Future revisions of the specification can further clarify vague sections or introduce new \gls{ec} for the key material.

As an evolution of the \gls{u2f} protocol, a central part of this thesis was the introduction of the \wa. By keeping the same secure concept of public-key cryptography that is abstracted from the user, the phishing resistance is ensured, too. A key difference and advancement are the division into two protocols, the low-level \gls{ctap} which defines the communication between the web browser (client) and the token and the high-level \gls{js} \gls{api}, the \wa{}. Furthermore, the \wa{} does not require that the token an external, i.e., roaming authenticator, but instead, it can be a built-in platform authenticator. This enabled the user to utilize already know and learned techniques, such as using their fingerprint sensor. Likewise, the \wa{} allows passwordless registration and authentication, which in contrast, the \gls{u2f} protocol did not allow. Given the fact that both protocols are standardized by either the \gls{w3c} and the \gls{itu}, the adoption rate of the \gls{api} is improved because the web browser vendors usually follow the recommendations and implement the functionality in a manageable time frame. 

\newpage

Additionally, the \wa{} was still in development and only a \gls{w3c} draft\footnote{The \gls{w3c} standardization process can be viewed in the \autoref{sec:w3c-process}} while finding the topic of this thesis, and the \gls{api} is only standardized since March 2019. This highlights the relevance of research on this topic. At the beginning of writing this thesis, Safari had only limited, opt-in support for the \wa, and iOS had no working solution at all. During the development of this work, Yubico in partnership with Brave announced and released a new variant of the YubiKey series to support iOS. Moreover, Apple released the stable release of Safari 13 with support for the \wa{} and Firefox for Android added support for the \gls{api}. This rapid development further strengthens the interest in this topic and the need to replace passwords. 

The questions that were defined at the beginning of this thesis in \autoref{sec:goals}, such as the threats of not using \gls{mfa}, problems that result from weak and re-used passwords, but also the question of whether \gls{mfa} can be made secure, have been answered.

Since the thesis focused on the use of \gls{mfa} solutions on the web, but at the same time delimitated concepts as OpenID Connect, OAuth 2.0 and \gls{sso}, further research can be done on these topics. For instance, \gls{sso} is an essential factor in enterprises, often used in conjunction with services and protocols such as \gls{ldap} or \gls{ad} from Microsoft. It has yet to be researched, whether the \wa{} is combinable with \gls{sso} providers.

In the consumer section, OAuth plays a relevant aspect as, e.g., Google, Facebook, or Microsoft allow the registration on a website with the account a user has on their site registered. Most recently, Apple announced support for OAuth with the Apple ID, too. A matter to clarify and study in this regard is the competition between the \wa{} and OAuth. In particular, the privacy aspect is interesting, because the providing party has the knowledge on which websites the user registered with their account. This allows extensive user tracking.\footcites[See][18]{fido-ct-2}[See][4]{osti_1257179}

One the one hand, emerging technologies such as \gls{iot} and Industry 4.0 are becoming more and more connected, but on the other hand, they increase the attack surface drastically. An open question for continued research is if the \wa{} is a suitable use case for secure and robust authentication which is done without user interaction, but instead by machines and computers autonomously. Further, it has to be evaluated if these, especially low-power computing, devices are capable of implementing both secure channels, such as \gls{tls} and calculations on \glspl{ec}.

\newpage

All in all, this thesis showed that internet users and their accounts are at risk because passwords are not suited for secure authentication in the way they are currently used. Weak passwords, password re-usage, and data breaches expose a constant threat. It is crucial to sensitize the user that these threats affect each and every one. Moreover, existing \gls{mfa} solutions are prone to phishing attacks since the way of transmission via \gls{sms} and e-mail is insecure and can be eavesdropped. Additionally, they are not resistant to social engineering attacks or malware on a user's device. \gls{u2f} and in particular the \wa{} are the only solutions to protect against phishing. While the \wa{} has the potential to (finally) replace passwords, it is not yet usable enough in all web browsers across different \glspl{os}. If the browser vendors implement the missing transportation protocols (\gls{ble} and \gls{nfc}) and if Apple adds support on iOS, this matter is solved though. A topic that is not well advertised, nor explained is the advice to use at least two authenticators to possess a backup method. With the help of the web browser vendors and a growing rate of adoption, the \wa{} is on the right way towards an internet with strong, secure, and simple authentication, even without passwords.