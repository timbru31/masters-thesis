\thispagestyle{noheader}
\setlength{\columnsep}{1cm}

\phantomsection
\pdfbookmark{Abstract}{abstract}

\begin{multicols}{2}
	\begin{large}
		\textbf{Abstract} \\ \\
	\end{large}
	Internet users are at constant risk, given that data breaches happen nearly daily. When a breached password is re-used, it renders their whole digital identity in danger. To counter these threats, the user can deploy additional security measures, e.g., the multi-factor authentication. This master's thesis introduces and compares the multi-factor authentication solutions one-time passwords, smartcards, security keys, and the Universal Second Factor protocol with a focus on their security. Further, the Web Authentication API is explained and compared with the other multi-factor authentication solutions. The outcome of this thesis is that multi-factor authentication is subject to phishing attacks but can be made phishing resistant with more efforts. In addition, the Web Authentication API has the potential to replace passwords, however it is not yet usable enough for the end user.
	\columnbreak \\
	\begin{large}
		\textbf{Kurzfassung} \\ \\
	\end{large}
	Internetnutzer sind einem ständigen Risiko ausgesetzt, da Sicherheitsbrüche fast täglich auftreten. Wenn ein gehacktes Passwort wiederverwendet wird, stellt dies eine Gefahr für die gesamte digitale Identität des Nutzers dar. Um diesen Bedrohungen entgegenzuwirken, kann der Benutzer zusätzliche Sicherheitsmaßnahmen, z.B. die Multi-Faktor Authentifizierung, einsetzen. Diese Masterarbeit stellt die Multi-Faktor Authentifizierungen Einmalpasswörter, Smartcards, Sicherheitsschlüssel und das Universal Second Factor Protokoll mit Fokus auf deren Sicherheit vor und vergleicht diese. Weiterhin wird die Web Authentication API erläutert und mit den anderen Multi-Faktor-Authentifizierungen verglichen. Das Ergebnis dieser Arbeit ist, dass die Multi-Faktor Authentifizierung trotzdem Phishing-Angriffen ausgesetzt ist, aber mit mehr Aufwand resistent gegenüber Phishing gemacht werden kann. Darüber hinaus hat die Web Authentication API das Potenzial Passwörter zu ersetzen, ist aber für den Endbenutzer noch nicht ausreichend nutzbar.
\end{multicols}

\addvspace{2cm}

\keywords{authentication, multi-factor authentication, mfa, two-factor authentication, 2fa, fido, fido2, web authentication api, webauth, webauthn, web-authentication}

\newpage