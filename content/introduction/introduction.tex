\chapter{Introduction}

\blindtext

\section{Methods of authentication}

There are multiple different methods or forms  respectively that can be used to authenticate a user against someone or something. Traditionally only knowledge, possession and trait are considered the different forms of authentication\footcite[See][140]{brotherston2017defensive}, but other sources also count methods like location- or time-based authentication\footcite{6296127}.  Therefore this thesis accounts them, too.\\
In the following sections each of those methods is described briefly.

\subsection{Knowledge}

The most common method of authentication is knowledge so therefore something the \frqq the user knows\flqq{}. Commonly used in web project are passwords. Other forms of knowledge are for example banking \gls{pin}, a passphrase, secret/recovery questions or a \gls{otp}. The security relies on the fact that the knowledge method is considered a secret that only the user knows.\footcite[See][467]{eckert-it-sec-9}

\subsection{Possession}

Another form of authentication is the possession, i.e. \frqq the user has\flqq{}. The most common example is a key for a lock. Other forms are for example a bank card or a security token, both realized in hardware and/or software.

\subsection{Biometrics}

Besides the knowledge and possession factors, another one is the biometrics. This factor is classified as something \frqq the user is\flqq{} and commonly includes the fingerprint, facial or iris scan. In theory many other characteristics like the gait, the ear, DNA or even the human odor could be used as a biometric factor.\footcite[See][30--34]{Jain2011}\\
These intrinsic factors are sometimes refereed to as traits or inherits, too.\footcite[See][186]{dasgupta2017multi}

While it seems naturally to authenticate a person with a biometric, it also comes with a couple of challenges. Both the \gls{frr}, i.e. a user is rejected even though it's a legitimate authentication attempt and \gls{far}, i.e. an imposter is granted access, need to be accounted. Compared to knowledge and possession factors the enrollment of the biometric and the continuous update of the sample is more complicated and expensive.\\
On the other hand though, it's more complicated to steal this factor than the others.

\subsection{Others/Location}

While the mentioned forms above are considered a standard in the literature, others forms exists, too. Those include e.g. the location of the user or a time-based authentication.\footcite[See][191]{dasgupta2017multi}

\section{Difference between authentication and verification and MFA vs 2SV}

One could argue that the sometimes the different authentication factors can be reduced to the same (\gls{otp} as something the user knows, since it relies on a secret).\\
In this case the authentication is technically correct not a \gls{mfa} but a multi-step verification, since the same factor is used multiple times.\\
Since this fine differentiation is not important for the thesis the term \gls{mfa} is used throughout.

\section{Differentiation of factors}

\subsection{Password}

Just knowledge. Often weak, re-used. Meant to be remembered. One factor only.\\
Protection by the server often not given, user's are writing it down etc.

\subsection{MFA}

More general term for 2FA. Can combine e.g. password with another method (like possession of hardware key, App) or trait (like TouchID, FaceID)

\subsubsection{OOB}

\gls{oob}

\subsection{WebAuth}

New API \gls{w3c}