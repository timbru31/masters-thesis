\chapter{Introduction}
\label{chapter:introduction}

\section{Problem Statement and Motivation}

\setlength\epigraphwidth{.5\textwidth}
\epigraph{\frqq Usernames and passwords are an idea that came out of 1970s mainframe architectures. They were not built for 2016.\flqq\footnotemark}{\textit{Alex Stamos}}
\footcitetext[See][]{stamos}

Passwords, in the way the are currently used, are not suited for the 21st century, like Alex Stamos, the former \gls{cso} of Facebook and Yahoo!, stated. The secure handling of passwords is a problem for many users. Passwords are re-used between different websites and often shared across private and work environments. This renders the (private) user data, but also business secrets at high risk.\\
To make things worse very few people are using \gls{mfa} nor a password manager in 2019. The majority of the users are either remembering their passwords or writing them down on a piece of paper - in cleartext.\footcites[See][]{ibm-security}[See][]{web-de-passwords}
At the same time the recorded amount of cybercrime cases is still increasing and for example phishing remains a big problem. While \gls{mfa} can protect against threats such as brute force attacks or stolen credentials, but they are still affected by phishing attacks. Besides that the SMS traffic is not considered secure anymore, yet it's used by a lot of \gls{mfa}, but the majority of the users is not using \gls{mfa}.\footcites[See][]{infratest-dimap}[See][6--7]{bka-cybercrime}

To counter this negative trends new \glspl{api} are emerging, for example the \wa. It's a standardized \gls{api} supported in major browsers such as Chrome, Firefox or Edge that allows  a secure registration, login and \gls{2fa} - all without the generation, storage and remembering of passwords by using asymmetric cryptography. The private keys are stored e.g. on external devices like USB sticks, but can be stored on built-in hardware, too and for example be protected by a fingerprint sensor.

\section{Goals of this thesis}

The goals of this thesis are an introduction into \gls{mfa} and the different authentication factors such as \frqq knowledge, possession and biometrics\flqq{} including the technical functionality, usability in web project and respectively web browser and their security risks alongside an introduction to the \wa. Those methods of authentication need to be mapped to actual forms of authentication such as passwords, security keys and fingerprint sensors, that need to be again evaluated securitywise.

The \wa{} and it's origin is being illustrated and technically in more depth explained. In this connection the question has to be answered if the \wa{} can increase the security and user comfort and usability. Of course the security and potentials risks of the \wa{} need to be taken into account.

Finally the thesis should answer the question if the \wa{} is ready to be used yet and whether it can replace passwords and existing \gls{mfa} solutions or be used in conjuction.
Besides that questions such as

\begin{itemize}
	\item What are the risks of not using \gls{mfa}?
	\item Why are weak password and re-usage such a big issue?
	\item Is there a protection against the weakest link, often being humans?
	\item If using \gls{mfa}, are there any risks, too?
	\item Are the architecture and algorithms used secure enough for usage in web projects and insecure connections?
	\item Is the \wa suitable and understandable for end users?
\end{itemize}

are taken into account and answered.

\section{Target audience}

The target audience of this thesis are technically experienced readers that have a good understanding of data security and privacy. Additionally the reader should have basic knowledge about the mathematics and functionality of algorithms like RSA, \gls{ecc} or symmetric and asymmetric key exchange (e.g. Diffie–Hellman key exchange). Furthermore the thesis is tailored towards interested (web) developers that want to understand the pros and cons of alternative registration and logins solutions, \gls{mfa} solutions and asymmetric cryptography and if the \wa{} suits their needs.

\section{Delimitation of this thesis}

Existing algorithms and concepts, as long as not required for the understanding of this thesis, are not explained in detail. It is not the goal of this thesis to perform a complete cryptanalysis, but to take other factors such as usability for the user, technical feasibility and web browser support into account. Different, but adjacent, technologies such as OAuth (2.0), OpenID Connect or \gls{sso} neither are a focus of this thesis.

\section{Approach and methodology}

Initially in chapter \ref{chapter:basics} the reader is introduced into the basics of authentication.\\
After that in the following chapters the areas

\begin{itemize}
	\item Single-Factor-Authentication
	\item \gls{mfa}
\end{itemize}

are explained, for example their technical functionality, and analyzed in regards of their security and potential risks and attacks such as phishing or \gls{mitm} attacks.

Hereupon the \wa{} is introduced in chapter \ref{chapter:webauth} and described in detail. The technical functionality is a key aspect of this chapter. Additionally the attacks the \wa{} can offer protection against are explained, but also asserted which security risks exists, too. Where suitable example source code listings are used to highlight these analysis.
% TODO or POC?

In the chapter \ref{chapter:comparison} the \wa{} is compared with existing \gls{mfa} solutions. Therefore it's reviewed if the \wa{} can be used in conjunction or as replacement for \gls{mfa}.

Concluding follows an evaluation based on the gained insights from the previous chapters with a conclusion and an outlook for further research and studies.
