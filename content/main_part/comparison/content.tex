\section{Comparison with Other Multi-Factor Authentications}
\label{chapter:comparison}

Because the \wa{} is an evaluation of the \gls{u2f} protocol the comparison of the \wa{} extends the comparison of \gls{u2f} to other \gls{mfa} solutions. They key advantage of the \wa{} in contrast to \glspl{otp} is the built-in phishing resistance. The specifications of the \gls{fido}2 project set the requirements for the protection against phishing attacks by verifying the \gls{rp}. This mitigations one of the biggest threats in other \gls{mfa} solutions.

In contrast to the \gls{u2f} protocol the \wa{} enables the possibility to use platform, i.e., built-in authenticators. This greatly enhances the usability and simplicity for the end user by providing already learned authentication flows, such as unlocking a device with a fingerprint. Built-in authenticators can in addition enable the protection by biometrics. This is possible due to prior works of, e.g., the \gls{uaf}.

 However, the current hurdles of the \wa{} is the missing interoperability of different transport protocols and the lack browser support for the Internet Explorer, iOS and many Android web browsers. In contrast, \gls{otp} work in these browsers, too. Also, the user is advised to keep a backup authenticator in case of theft or loss of their primary authenticator. Nonetheless, this behavior is identical to \gls{otp} where the user has the possibility to save backup codes.\footcites[See][36]{10.1007/978-3-319-45931-8_3}
 
An unweighted fact is the current rate of adaption. Since the \wa was certified in March 2019, very few services have implemented the \gls{api} - in comparison with \gls{mfa} that are in use since five to ten years.\footcites[See][]{works-with-yubico}
 
Regardless, the \wa{} has the potential to obsolete the necessity of a second-factor by replacing traditional passwords with public-key cryptography. Moreover, the user does not need to worry about remembering passwords anymore because they can use their biometrics to unlock for instance a platform authenticator. No other introduced \gls{mfa} solution has the potential to achieve the same.
