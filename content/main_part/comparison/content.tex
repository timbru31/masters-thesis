\section{Comparison with Other Multi-Factor Authentications}
\label{chapter:comparison}

Because the \wa{} is an evaluation of the \gls{u2f} protocol, the comparison of the \wa{} extends the comparison of \gls{u2f} to other \gls{mfa} solutions. The key advantage of the \wa{} in contrast to \glspl{otp} is the built-in phishing resistance. The specifications of the \gls{fido}2 project set the requirements for the protection against phishing attacks by verifying the \gls{rp}. This solves one of the biggest threats in other \gls{mfa} solutions.

In contrast to the \gls{u2f} protocol, the \wa{} enables the possibility to use a platform, i.e., built-in authenticators. This greatly enhances the usability and simplicity for the end user by enabling already learned authentication flows, such as unlocking a device with a fingerprint for the registration and login on the web. Built-in authenticators can, additionally, enable protection by biometrics. This is possible due to prior works of, e.g., the \gls{uaf}.

 However, the current hurdles of the \wa{} are the missing interoperability of different transport protocols and the lack of web browser support for Internet Explorer, iOS, and many Android web browsers. In contrast, \glspl{otp} work in these web browsers, too, and do not require an update of the web browser. Also, the user is advised to keep a backup authenticator in case of theft or loss of their primary authenticator, but this essentials aspect is not well highlighted and advertised. Nonetheless, this behavior is identical to \gls{otp}, where the user has the possibility to save backup codes to prevent an account lockout.\footcites[See][36]{10.1007/978-3-319-45931-8_3}
 
 Additionally, the \wa{} can not protect against an account takeover if the system itself is, for example, infected by malware. Malware is still able to steal a user's session, which can lead to further attacks. Out of scope for the \wa{} is transportation, which is defined by the \gls{ctap}. Malware can access the Bluetooth or \gls{usb} interfaces and intercept the transmissions. However, the private key material never leaves the authenticator. As long as the secure storage of the authenticator is tamper-resistant, the private keys are not at risk. In contrast, malware for smartphones might be able to steal the secret used to generate \glspl{otp} if it is not stored in a secure manner.
 
An unweighted fact is the current rate of adaption. Since the \wa{} was certified in March 2019, very few services, have implemented the \gls{api} - in comparison with \gls{mfa} solutions, such as \glspl{otp} and RSA SecurID, that are in use for five to ten years.\footcites[See][]{works-with-yubico}
 
Regardless, the \wa{} has the potential to obsolete the necessity of a second-factor at all by replacing traditional passwords with public-key cryptography that are secure, unique, and phishing resistant for every \gls{rp} the user registers with. Moreover, the user does not need to worry about remembering passwords anymore because they can use their biometrics to unlock, for instance, a platform authenticator. No other introduced \gls{mfa} solution has the potential to achieve the same.
