\chapter{Basics of authentication}
\label{chapter:basics}

\section{Methods of authentication}
There are multiple different methods or forms, respectively, that can be used to authenticate a user against someone or something. Traditionally only knowledge, possession, and trait are considered the different forms of authentication,\footcites[See][299]{10.2307/27845364}[See][140]{brotherston2017defensive}[][47]{anderson2008security} but other sources also introduce or take new methods into account such as the location- or time-based authentication.\footcites[][]{6296127}[See][191]{dasgupta2017multi} Therefore, this thesis accounts for them, too, and describes the methods in the following sections briefly and a detailed analysis of the security, potential risk, and threats is done in \autoref{one-factor-threats}.

\subsection{Knowledge}

The most common method of authentication is knowledge, i.e., \frqq something the user knows\flqq{}. Commonly used in \gls{it} are passwords. Other forms of knowledge are, for example, a \gls{pin} used, e.g., banking (ATM's, credit cards) or telephony (SIM card), a passphrase, secret, or recovery questions or a \gls{otp}. The security relies on the fact that the knowledge method is considered a secret that only the user knows.\footcite[See][467]{eckert-it-sec-9}

\subsection{Possession}

Another form of authentication is the possession, i.e., \frqq something the user has\flqq{} (physically). The most basic example is a key for a lock. Other forms are, for example, a bank, or ID card that can use techniques like \gls{rfid}, an onboard chip or magnetic stripes to store the information. In \gls{it} security tokens are often used, which can be a hardware (such as a YubiKey, a RSA SecurID or a smart card) or software (e.g., a smartphone application) token. They can either be disconnected, connected (e.g., via USB or as a smart card) or contactless (e.g., via \gls{nfc}, \gls{ble} or \gls{rfid}). Sometimes these tokens contain a display itself that can show further information.\footcites[See][24]{265831}[][]{Dressel:2019:SZT:3319499.3328225}[See][8--11]{1698485}

The security of this method relies on the fact that only a legitimate user has access to the possession factor and that no intruder has access or can, for example, make a copy. The most significant risk is the loss or theft of the possession, although, for instance, some security tokens are itself protected with authentication like the fingerprint of the legitimate user.

\subsection{Biometrics}

Besides the knowledge and possession factors, another one is biometrics. This factor is classified as \frqq something the user is\flqq{} and commonly includes the fingerprint, facial, or iris scan. In theory, many other characteristics like the gait, the ear, DNA, or even the human odor could be a biometric factor.\footcite[See][30--34]{Jain2011}\\
These intrinsic factors are sometimes referred to as traits or inherit, too.\footcite[See][186]{dasgupta2017multi}

While it seems natural to authenticate a person with a biometric, it also comes with a couple of challenges. Both, the \gls{frr}, i.e., the system rejects a user even though it is a legitimate authentication attempt and \gls{far}, i.e., an imposter is granted access, need to be accounted for the usage. Compared to knowledge and possession factors, the enrollment of the biometrics and the continuous update of the sample is more complicated and expensive.\footcites[See][18--24]{Jain2011}[See][34--37]{265831}

On the other hand, it is more complicated to steal, share, or copy this factor than the others - but also nearly impossible to replace a compromised biometrics. The usability varies because of the quality of the used biometrics module, the chosen biometrics itself, and the availability of the biometrics.

\begin{figure}[hbt]
	\centering
	\includesvg[width=\textwidth,pretex=\relscale{0.5}]{pics/svg/biometrics_auth_flow}
	\caption{}
	\label{fig:biometrics_auth_flow}
\end{figure}

\subsection{Further methods of authentication}

While the mentioned authentication forms above are considered a standard in the literature, other forms exist, too. Those include, for example, the location of the user. The location-based approach grants or denies access based on the current location. The location can either be physical (e.g., via GPS) or digital with, e.g., an IP address.\footcite{6296127}

Another form is time-based authentication. A typical example is time-limited access to a banking safe, which can only be opened at specific times of the day, a time lock secures it. In \gls{it} this form of authentication helps to protect against, for instance, phishing attacks from abroad, because the access is granted or denied based on the time and usual time routines where, for instance, a user logs typically on.\footcite[See][191]{dasgupta2017multi}

\section{Wording differences between multi-factor, multi-step, authentication, and verification}

The naming of the chosen authentication or verification methods by companies is often confusing or difficult to understand. The terms used by companies vary from \gls{2fa}, often just calling it \gls{2fa},\footcite{apple_2fa} to two-step-verification, sometimes written as 2-Step Verification, too.\footcites[][]{apple_s2v}[][]{playstation}[][]{google_2-step_verification}[][]{microsoft_2sv}

One could argue that the different authentication factors can be reduced to a single one, e.g., that an \gls{otp} is \frqq something the user knows\flqq{} since it relies on a secret that \textit{could}, in theory, be memorized, too.\\
In this case, the term \gls{mfa} or 2FA is technically incorrect, since it is instead a multi-step authentication because the same factor is used multiple times. However, it has to be noted that using the same authentication factor multiple times is weaker than using different authentication factors.\footcite[See][117]{grimes2017hacking}

Besides that, (user) verification is a part of the authentication process this little differentiation of verification vs. authentication and multi-step vs. multi-factor is not crucial for this thesis, and the term \gls{mfa} is used throughout.

% TODO Check this sections
%\section{Differentiation of factors // Introduction to 1FA, MFA and WebAuthn}
%
%As the main focus of this master's thesis is to evaluate \gls{mfa} with the Web Authentication API the following sections briefly introduces the common authentication methods ranging from single factor authentication with password to \gls{mfa} to the WebAuth.
%
%\subsection{MFA}
%
%More general term for 2FA. Can combine e.g., password with another method (like possession of hardware key, App) or trait (like TouchID, FaceID)
%
%\subsubsection{OOB}
%
%\gls{oob}
%
%\subsection{WebAuth}
%
%New API \gls{w3c}