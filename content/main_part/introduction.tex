\chapter{Basics of authentication}
\label{chapter:basics}

\section{Methods of authentication}
There are multiple different methods or forms, respectively, that can be used to authenticate a user against someone or something. Traditionally only knowledge, possession, and trait are considered the different forms of authentication,\footcites[See][299]{10.2307/27845364}[See][140]{brotherston2017defensive}[][47]{anderson2008security} but other sources also introduce or take new methods into account such as the location- or time-based authentication.\footcites[][]{6296127}[See][191]{dasgupta2017multi} Therefore, this thesis accounts for them, too, and describes the methods in the following sections briefly including a diagram of an example authentication flow. A detailed analysis of the security, potential risk, and threats follows in \autoref{sec:one-factor-threats}

\subsection{Knowledge}

The most common method of authentication is knowledge, i.e., \frqq something the user knows\flqq{}. Commonly used in \gls{it} are passwords. Other forms of knowledge are, for example, \glspl{pin}, passphrases, secrets, recovery questions, or \glspl{otp}. The \gls{pin} is a good example for usage, e.g., in banking (ATM's, credit cards) or telephony (SIM cards). The security relies on the fact that the knowledge method is considered a secret that only the user knows. When compromised it is relatively easy to replace the knowledge with a different secret the user knows. Unintentional side effects are that the user may has to replace the used knowledge everywhere in case of re-use.\footcite[See][467]{eckert-it-sec-9}

\newpage

\begin{figure}[hbt]
	\centering
	\includesvg[width=\textwidth,pretex=\relscale{0.5}]{pics/svg/knowledge_auth_flow}
	\caption[Exemplary, but simplified, authentication by knowledge flow]{Exemplary, but simplified, authentication by knowledge flow\footnotemark}
	\label{fig:knowledge_auth_flow}
\end{figure}
\footnotetext{Source: diagram by author.}

\autoref{fig:knowledge_auth_flow} shows a simplified authentication by knowledge flow. First the user visits in this example a website and enters their password in the corresponding form fields. When the user submits the form, the transferred password is often transformed, e.g. hashed and salted. If the user is known in the database, the stored (hash of) the password is retrieved and then compared to the given one. Only if the hashes are identical, then the login succeeds. Otherwise it fails. The \frqq access denied/cancel\flqq{} and \frqq checkmark\flqq{} symbols are chosen, since it can't be verified if the authentication is made by the geniuine user or an imposter that gained access to the knowledge of the attacked user, in this case their password.

\subsection{Possession}

Another form of authentication is the possession, i.e., \frqq something the user has\flqq{} (physically). The most basic example is a key for a lock. Other forms are, for example, a bank, or ID card that can use techniques like \gls{rfid}, an onboard chip or magnetic stripes to store the information. In \gls{it} security tokens are often used, which can be a hardware (such as a YubiKey, a RSA SecurID or a smart card) or software (e.g., a smartphone application) token. They can either be disconnected, connected (e.g., via USB or as a smart card) or contactless (e.g., via \gls{nfc}, \gls{ble} or \gls{rfid}). Sometimes these tokens contain a display itself that can show further information.\footcites[See][24]{265831}[][]{Dressel:2019:SZT:3319499.3328225}[See][8--11]{1698485}

\begin{figure}[hbt]
	\centering
	\includesvg[width=\textwidth,pretex=\relscale{0.5}]{pics/svg/possession_auth_flow}
	\caption[Exemplary, but simplified, authentication by possession flow]{Exemplary, but simplified, authentication by possession flow\footnotemark}
	\label{fig:possesion_auth_flow}
\end{figure}
\footnotetext{Source: diagram by author}

\autoref{fig:possesion_auth_flow} shows an example authentiaion flow with a smartcard. First, the user inserts the given smartcard into their computer. The data is read subsequently. Contemporaneous the application or sytem reads the stored database entry and compares the data to the one stored on the smartcard. If the data is equal or matches and the user authroized, then the authentication succeeds. Again any user can log on as long as they are in the possession of the smartcard.

\subsection{Biometrics}

Besides the knowledge and possession factors, another one is biometrics. This factor is classified as \frqq something the user is\flqq{} and commonly includes the fingerprint, facial, or iris scan. In theory, many other characteristics like the gait, the ear, DNA, or even the human odor could be a biometric factor.\footcite[See][30--34]{Jain2011}\\
These intrinsic factors are sometimes referred to as traits or inherit, too.\footcite[See][186]{dasgupta2017multi}

While it seems natural to authenticate a person with a biometric, it also comes with a couple of challenges. Both, the \gls{frr}, i.e., the system rejects a user even though it is a legitimate authentication attempt and \gls{far}, i.e., an imposter is granted access, need to be accounted for the usage. Compared to knowledge and possession factors, the enrollment of the biometrics and the continuous update of the sample is more complicated and expensive.\footcites[See][18--24]{Jain2011}[See][34--37]{265831}

On the other hand, it is more complicated to steal, share, or copy this factor than the others - but also nearly impossible to replace a compromised biometrics. The usability varies because of the quality of the used biometrics module, the chosen biometrics itself, and the availability of the biometrics.

\begin{figure}[hbt]
	\centering
	\includesvg[width=\textwidth,pretex=\relscale{0.5}]{pics/svg/biometrics_auth_flow}
	\caption[Exemplary, but simplified, authentication by biometrics flow]{Exemplary, but simplified, authentication by biometrics flow\footnotemark}
	\label{fig:biometrics_auth_flow}
\end{figure}
\footcitetext[Source: diagram by author, based on][11]{Jain2011}


\autoref{fig:biometrics_auth_flow} shows an exemplary authentication flow using biometrics, in the case with a fingerprint. First, the user interacts with the sensor that reads the fingerprint and extracts the biometric template. Generally, the template is then transformed into a more comparable format. For instance, fingerprints are scanned for minutiae and their direction. Simultaneously, the system retrieves the stored fingerprint or searches for it. The system now compares the stored probe to the fresh one. A threshold value that determines how much difference is tolerable decides finally if the authentication attempt can proceed or has to be aborted and denied. If the authentication succeeds, the stored template can be updated in the database, as denoted by the dotted grey arrow.

\subsection{Further methods of authentication}

While the mentioned authentication forms above are considered a standard in the literature, other forms exist, too. Those include, for example, the location of the user. The location-based approach grants or denies access based on the current location. The location can either be physical (e.g., via GPS) or digital with, e.g., an IP address.\footcites{6296127}[See][Chapter 13.9]{2308830}

Another form is time-based authentication. A typical example is time-limited access to a banking safe, which can only be opened at specific times of the day, a time lock secures it. In \gls{it} this form of authentication helps to protect against, for instance, phishing attacks from abroad, because the access is granted or denied based on the time and usual time routines where, for instance, a user logs typically on.\footcite[See][191]{dasgupta2017multi}

Further methods of authentication are for example the social authentication, also referred as \frqq someone the user knows\flqq. For example Facebook uses this method to ensure the authentication is attempt is genuine by asking the user to identify a set of their friends. Of course the social authentication works in other scenarios, especially offline, too.\footcites[See][]{Brainard2006}[See][278--279]{shostack2014threat} Besides these methods \frqq something the user does\flqq{} is another form of authentication. Examples range from keystrokes to online shopping behavior.\footcites[See][]{10.1007/978-3-642-18178-8_9}[See][]{7460349}

\section{Processes of authentication}

The process of authentication can be done in three different manners that are explained in the following subsections. These are namely:

\begin{enumerate}
	\item \textbf{Active authentication}, where a user has to initiate the process
	\item \textbf{Passive authentication}, where the user does not need to interact with the system
	\item \textbf{Continuous authentication}, where a system constantly monitors and authenticates the user
\end{enumerate}

\subsection{Active authentication}
\label{subsec:active_auth}

The most common process of authentication is the active authentication. In this process of authentication, the user has to initiate the authentication, e.g., by opening the website and entering their password in the form fields, by pressing a button or placing their fingerprint on the corresponding sensor.\footcites[See][185--186]{10.1007/978-3-319-05452-0_14}

\subsection{Passive authentication}

In contrast to the active authentication process, the user is in the passive authentication process authenticated without an action on their part. Use case are, for \gls{rfid} chips that constantly send a signal in a short range and can open a door when the user approaches it. Further examples can be the analysis of keystroke or touch screen usage patterns. In comparison with active authentication this process is more low-friction.\footcites[See][186]{10.1007/978-3-319-05452-0_14}[See][]{185306}

\subsection{Continuous authentication}

Further, the process of continuous authentication exists. In this case the user is constantly authenticated or monitored, to ensure that it it still the originally authenticated user whom is using the system. It is important that the authentication happens in a non-intrusiveness way. Commonly used for continuous authentication are biometrics, such as fingerprint, facial recognition or keystroke patterns.\footcites[See][236--238]{dasgupta2017multi}[See][]{7444124}

Unfortunately the term active authentication is often used to describe continuous authentication, too. To avoid confusion the solely term continuous authentication is used to refer to this process of authentication, while any mentions of active authentication refers to the process described in \autoref{subsec:active_auth}.

\newpage

\section{Wording differences between multi-factor, multi-step, authentication, and verification}

The naming of the chosen authentication or verification methods by companies is often confusing or difficult to understand. The terms used by companies vary from \gls{2fa}, often just calling it \gls{2fa},\footcite{apple_2fa} to two-step-verification, sometimes written as 2-Step Verification, too.\footcites[][]{apple_s2v}[][]{playstation}[][]{google_2-step_verification}[][]{microsoft_2sv}

One could argue that the different authentication factors can be reduced to a single one, e.g., that an \gls{otp} is \frqq something the user knows\flqq{} since it relies on a secret that \textit{could}, in theory, be memorized, too, but practically is not memorizable.\\
In this case, the term \gls{mfa} or 2FA is technically incorrect, since it is instead a multi-step authentication because the same factor is used multiple times. However, it has to be noted that using the same authentication factor multiple times is weaker than using different authentication factors.\footcite[See][117]{grimes2017hacking}

Besides that, (user) verification is a part of the authentication process this little differentiation of verification vs. authentication and multi-step vs. multi-factor is not crucial for this thesis, and the term \gls{mfa} is used throughout.

% TODO Check this sections
%\section{Differentiation of factors // Introduction to 1FA, MFA and WebAuthn}
%
%As the main focus of this master's thesis is to evaluate \gls{mfa} with the Web Authentication API the following sections briefly introduces the common authentication methods ranging from single factor authentication with password to \gls{mfa} to the WebAuth.
%
%\subsection{MFA}
%
%More general term for 2FA. Can combine e.g., password with another method (like possession of hardware key, App) or trait (like TouchID, FaceID)
%
%\subsubsection{OOB}
%
%\gls{oob}
%
%\subsection{WebAuth}
%
%New API \gls{w3c}