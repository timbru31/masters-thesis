\subsection{Algorithm}

As both the \gls{hotp} and the \gls{totp} are based on the \gls{hmac} specification, the underlying algorithm needs to be evaluated first. The vital factor is the chosen cryptographic hash algorithm. Mostly \gls{sha}-1 is used since it is the default defined in the \gls{rfc}.\footcite[See][3]{m2011rfc}

Given that both \gls{sha}-1 and MD5 are considered insecure, one has to ask if they are still considered secure in the \gls{otp} context. Because of the truncation and limited character set, the collision resistance of the chosen cryptographic hash algorithm is not of importance for the security of the \gls{otp} generation. Therefore, using the \gls{md}5 and \gls{sha}-1 algorithms do not expose a threat. Besides, both the \gls{rfc} and the \gls{bsi} still list these algorithms as secure for \gls{hmac} after a consideration of the collision attacks.\footcites[See][]{10.1007/978-3-319-63688-7_19}[See][18]{bsi2019recommendations}[See][2]{s2011rfc}[See][395]{eckert-it-sec-9}

It is more important to implement the algorithm correctly than replacing the used hash function. In the past, e.g., Google did not issue \gls{otp} values with a leading zero. Besides that, the defined minimum length of the \gls{otp} values is six digits. Meanwhile, the \gls{rfc} supports up to ten. However, nearly no service provider uses more than six digits. This decreases the \gls{otp} entropy and strengthens the brute-force attack.\footcites[See][369]{10.1007/978-3-662-45472-5_24}

Further, for example, the online gaming platform Steam uses a different alphabet and character length. These divergences show that not all implementing parties follow the recommendations of the \gls{rfc}. Moreover, the user cannot verify that the algorithms are correctly implemented. Because \gls{hotp} and \gls{totp} rely on a shared secret, it is crucial that both the server and client store the secret in a secure manner.\footcites[See][6--7]{steam-totp}[See][11--13]{m2005rfc}

A theoretical vulnerability is to use the look-ahead window feature. It enables an attacker to use a token that is much longer valid than it should be. The larger the look-ahead window period is, the bigger the time-frame an attack has brute-force, or phish is, too. Also, it is essential that the \gls{otp} is invalidated after a successful use or when the time in the look-ahead window has passed.\footcites[See][369]{10.1007/978-3-662-45472-5_24}[See][11]{m2005rfc}

In addition, the \glspl{otp} are subject to a brute-force attack. The server must throttle the number of tries a user can make to counter this attack.\footcites[See][6]{m2011rfc}[See][21-22]{m2005rfc}[See][240]{Schwartz2018}

The main threat, however, remains the vulnerability to phishing because \glspl{rfc} do not specify any requirements or recommendations on how to verify the origin a user communicates with. The phishing threat is examined in more depth in the following \autoref{subsec:transport}.

\subsection{Transportation and Generation}
\label{subsec:transport}

Given that the fact that the weaknesses of both \gls{sha}-1 and \gls{md}5 do not expose a threat to the use of \gls{otp}. Instead, the transmission and secure generation of the \glspl{otp} pose a challenge. This section considers the transportation mediums \gls{sms}, e-mail, and the threats in regards to a generation by smartphone apps.

\subsubsection{SMS}

The most significant advantage of \gls{sms} as a transportation medium is every mobile, ranging from an old Nokia 3310 to a new iPhone 11 Pro, is capable of receiving \gls{sms}. All major mobile phone \glspl{os} come with an \gls{sms} application pre-installed, so no external apps are required. The first \gls{sms} was sent in 1994, and while the \gls{sms} traffic is decreasing, there were 9 billion messages sent solely in Germany in 2018.\footcites[See][2--3]{alpert2012mobile}[See][57]{bundesnetzagentur}

Although there are some significant advantages with \gls{sms} transportation, such as the easy to use factor and the fact that the user does not need to know the \gls{otp} secret, it also comes with many downsides. Besides the cost aspect of \gls{sms} traffic, both for the sender and potentially for the receiver due to roaming fees, the current state of \gls{sms} traffic is considered insecure.\footcites[See][167]{8632643}

The \gls{sms} traffic relies on the \gls{ss7} network, which was developed in the 1970s. It has multiple security flaws that allows an attacker to eavesdrop or modify the in- and out-coming traffic.\footcites[See][17--18]{WELCH201717}[See][3--4]{7997246}[See][40, 46]{puzankov2017stealthy}

\autoref{fig:2fa_flow_ss7} shows the described \gls{mfa} flow of \autoref{fig:2fa_flow} using \gls{totp}. In this scenario, the attacker is able to phish the \gls{totp} designated for the user. The figure shows that the attacker uses an exploit in the \gls{ss7} network. This allows them to intercept all incoming \gls{sms}. With, e.g., a phished password, the attacker bypasses the enabled \gls{2fa} without the user's knowledge.

\begin{figure}[hbt]
	\centering
	\includesvg[width=\textwidth,pretex=\relscale{1}]{pics/svg/2fa_flow_ss7}
	\caption[\gls{ss7} exploit to phish an \gls{otp} used in \gls{mfa}]{\gls{ss7} exploit to phish an \gls{otp} used in \gls{mfa}\footnotemark}
	\label{fig:2fa_flow_ss7}
\end{figure}
\footnotetext{Source: diagram by author}

Another negative aspect of \gls{sms} transportation is routing. Many companies rely on third-party providers to send \gls{sms} to the user. These providers are using countries where \gls{sms} are very cheap, or route them the cheapest way, but on the other hand, the \gls{ss7} security measures such as \gls{sms} home routing and not enforced. This results in a higher security risk that the \gls{sms} is compromised while delivered to the user. Also, third-party providers are given access to the \gls{otp}, which enables the risk of a malicious insider.\footcites[See][153]{10.1007/978-3-642-39235-1_9}[See][4, 9, 12]{Certic2018}

In contrast to the web and e-mail, the user is not aware of phishing attacks in the \gls{sms} context. However, studies show that a new technique called the \gls{vcfa} is already in use. In this scenario, the attacker sends the victim a (spoofed) \gls{sms} and impersonates the service provider. They tell the user that, e.g., fraudulent access was detected and in order to block this attempt, the user needs to reply with the \gls{otp} code for security measure.\footcites[See][6--7]{JAKOBSSON20186}[See][4--5]{SIADATI201714}

\newpage

\autoref{fig:2fa_flow_forward_phishing} shows an example of a \gls{vcfa}. An attack logs in to the user's account with, e.g., hacked or phished credentials. Because \gls{mfa} protects the account, the user receives verification code via \gls{sms} or smartphone. The attacker now sends a fake \gls{sms} to the victim, stating that the service has detected unusual activity. In order to block this attempt and prove that the user is the legitimate account owner, they should reply with the just received verification code. Of course, the attacker now has access to the \gls{otp}, too, since they convinced the user into forwarding their code.\footcites[See][66]{10.1007/978-3-319-29938-9_5}

\begin{figure}[hbt]
	\centering
	\includesvg[width=\textwidth,pretex=\relscale{1}]{pics/svg/2fa_flow_forward_phishing}
	\caption[\gls{vcfa} to phish an \gls{otp} used in \gls{mfa}]{\gls{vcfa} to phish an \gls{otp} used in \gls{mfa}\footnotemark}
	\label{fig:2fa_flow_forward_phishing}
\end{figure}
\footnotetext{Source: diagram by author}

Another threat is specialized malware for mobile phones. Especially for Android, there exist multiple \gls{sms} trojans that are capable of intercepting the \gls{sms}, too. Some of them even target specific banking apps. Additionally, apps disguise as a useful application or are a repackaged legitimate app with a backdoor.\footcites[See][146--149]{dmitrienko2014security}[See][152--154]{10.1007/978-3-642-39235-1_9}[See][114]{HAMED2017109}[See][]{eset-bypass2fa}

\newpage

Moreover, social engineering attacks that target the mobile service operator are an attack vector. \autoref{fig:2fa_flow_sim_hack} shows an \gls{mfa} flow using \gls{totp}, but in this case, with another phishing scenario that targets the service provider. An attacker has again access to the user's password, e.g., from a previous, successful phishing attack. In order to obtain or phish, respectively, they target the human weakness in the cell phone provider of the user. They successfully convince them to activate another \gls{sim} card for the victim's phone number and receive the \gls{sms} with the \gls{totp}, too, which enables the attacker to complete the \gls{mfa} flow successfully. This type of attack is also called \gls{sim} swap scam or fraud. In the recent past, Twitter's CEO was a victim of such an attack that lead to an account takeover.\footcites[See][19]{BLAICH201719}[See][]{twitter-hack}[See][74--76]{10.1007/978-3-642-19228-9_7}
\\
\begin{figure}[hbt]
	\centering
	\includesvg[width=\textwidth,pretex=\relscale{1}]{pics/svg/2fa_flow_sim_hack}
	\caption[Social engineering used to phish an \gls{otp} in \gls{mfa}]{Social engineering used to phish an \gls{otp} in \gls{mfa}\footnotemark}
	\label{fig:2fa_flow_sim_hack}
\end{figure}
\footnotetext{Source: diagram by author}

Another variant that is technically more complex, but feasible, is the \gls{sim} card cloning. This allows the attacker to intercept the \gls{totp}, too. Even if the phone number cannot be registered twice, it still enables the eavesdropping.\footcites[See][873]{eckert-it-sec-9}[See][11--12]{sim-clone}

Given all these facts, \gls{sms} transportation should be avoided for the usage with \glspl{otp}, since there are multiple flaws in the \gls{ss7} network itself and the process of how the \gls{sms} reaches the user. It is also not resistant against phishing or mobile phone trojans. Both the \gls{bsi} and the \gls{nist} advise service providers not to use \gls{sms} for the transmission of \glspl{otp} and \glspl{tan} anymore.\footcites[See][8]{JAKOBSSON20186}[See][27]{bsi2019recommendations2}[See][19]{SP80063B}[See][407]{10.1007/978-3-662-54970-4_24}

Further, it cannot be guaranteed that the user has a working mobile network, that the registered mobile phone number is still active, or that the user receives the \gls{sms} in time. These non-influenceable, external factors strengthen the fact that \gls{sms} is not a secure and reliable choice for the transportation medium of \glspl{otp}.\footcites[See][327]{5945255}

\subsubsection{App}
\label{sec:app}
 
 In contrast to the transportation of the \gls{otp} via \gls{sms}, using a standalone app such as Google Authenticator, Authy, or even the mobile \gls{otp} app for Java-based phones offer some advantages. While the user has to be connected to the cellular network when receiving the \gls{otp} via \gls{sms}, the app solution works in offline or bad network connectivity use cases, too.\footcites[See][228]{6920371}
 
 Furthermore, the app solution is cheaper because no transaction fee has to be paid by the sender or receiver. It also solves the roaming problem. On the other hand, the app needs to be maintained and updated to protect against, e.g., vulnerabilities in third-party libraries and to ensure compatibility with future devices and \gls{os} versions.
 
 The setup of the \gls{otp} is not phishing resistant either. A malware on, e.g., the desktop can intercept the shared secret, or the mobile phone can contain malware. This malware can intercept, for instance, the camera, or later access the generated \gls{otp}, or access the app's database where the secrets are stored.\footcites[See][371]{10.1007/978-3-662-45472-5_24}[See][407]{10.1007/978-3-662-54970-4_24}
 
 \autoref{fig:2fa_flow_malware} shows the scenario where an attacker successfully infected a smartphone of a user with mobile malware. The hacker now has the possibility to access either the \gls{otp} secret or to forward the secret when a user opens the app. This enables an attacker to bypass \gls{2fa} if, e.g., beforehand, the password was phished or obtained from another data breach.

 \newpage
 
 \begin{figure}[hbt]
	\centering
	\includesvg[width=\textwidth,pretex=\relscale{1}]{pics/svg/2fa_flow_malware}
	\caption[Mobile malware used to phish an \gls{otp} in \gls{mfa}]{Mobile malware used to phish an \gls{otp} in \gls{mfa}\footnotemark}
	\label{fig:2fa_flow_malware}
\end{figure}
\footnotetext{Source: diagram by author}
 
 Besides that, the app itself can contain vulnerabilities. For instance, the \gls{2fa} app Authy suffered from a vulnerable backend that could be exploited to bypass the \gls{2fa}. Additionally, the user has to ask if the app is legitimate and from a trustworthy source and needs to have faith in the app that it keeps their data safe.\footcites[See][]{eset-bypass2fa}[See][]{sakurity-authy}
 
\subsubsection{E-Mail}

Another widely used form to transmit the \gls{otp} from the server to the user is the distribution via e-mail. E-mails are by comparison with apps more accepted by the users and do not require a user to provide their cellphone number, but rather only the e-mail address that is mostly provided to the service nevertheless.

However, e-mail traffic comes with threats, too. In the first place, unencrypted e-mail traffic can be intercepted by a \gls{mitm}, therefore exposing the \gls{otp}. For example, in August 2019, only 90\% of Google Mail's outgoing traffic was encrypted. Malware on the desktop or smartphone can intercept incoming e-mail and even delete the message without the users' knowledge, the \autoref{fig:2fa_flow_malware} applies here, too, with the variation that the \gls{otp} is not generated on the device but instead only received.\footcites[See][]{email-encryption}

Besides that, e-mails re-introduce the problem of delayed reception of the \gls{otp}. Especially techniques such as grey-listing delay incoming messages to avoid spam. Examples such as network connectivity problems, dedicated attack (\gls{ddos}) or exceeded e-mail storage quota further increase the potential of unreliable \gls{otp} transportation.\footcites[See][]{rfc6647}

Unfortunately, e-mails are not phishing resistant, either. The same threats of \gls{sms} apply, too. A malware, both on the desktop or mobile, can intercept and forward a received \gls{otp} to an attacker.
