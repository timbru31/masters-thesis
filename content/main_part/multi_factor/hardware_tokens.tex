\section{Security Tokens}

% FIPS 140-2?
Besides smartcards, further \gls{mfa} soltuions with possession as the second factor involve the usage of security tokens.

Hardware and software are possible. The minimum security specification are, e.g., defined in \gls{nist} \gls{fips} 140-3.\footcites[See][]{FIPS140-3}

\subsection{RSA SecurID}

The RSA SecurID exists in several variants, both hardware and software token. First hardware revisions used a 64-bit proprietary protocol called \flqq SecudID has function\flqq, hardware keys newer than 2003 use the standardized 128-bit RSA algorithm in order to generate \glspl{otp}. Newer revisions also feature a USB port which allows the device to store custom certificates, too, i.e., making it a smartcard device, too. Each token contains a burned in seed, a random key that was generated while manufacturing the device. Since this seed needs to be known to validate the \gls{otp}, the RSA SecurID server needs to be used.\footcites[See][479--480]{eckert-it-sec-9}

Mobile application exists, too, offering support for a soft-token based solution.

\subsection{YubiKey}

Variety of tokens available, all support Yubico's \gls{otp} algorithm, \gls{hotp} and \gls{totp}. Many support \gls{u2f}, too. Some static passwords and PKI. Some are \gls{piv} conform and \gls{fips} 140 certified.

\subsection{Software tokens}