\section{Security Tokens}

Besides smartcards, further \gls{mfa} solutions with possession as an additional factor are security tokens or keys. These security tokens exist as pure hardware solutions, as well as software based solutions. The minimum security requirements for the cryptographic modules are define in, e.g., the \gls{nist} \gls{fips} 140-3 standard.\footcites[See][]{FIPS140-3} This section introduces the well known security tokens \frqq RSA SecurID\flqq{} and \frqq YubiKeys\flqq. Typically security tokens either store a private key used in public-key cryptography or the shared secret in order to generate or validate \glspl{otp}.\footcites[See][Chapter 28.4.3]{1174011}

As many security tokens support the \gls{u2f}, these security tokens are not part of this section. Instead, the underlying concepts of the \gls{u2f} \gls{api} are explained and analyzed in \autoref{chapter:webauth}, since the \wa{} originated from the \gls{u2f} specification.

\subsection{RSA SecurID}

The RSA SecurID exists in several variants, both hardware and software token. First hardware revisions used a 64-bit proprietary protocol called \frqq SecurID hash function\flqq. Hardware keys newer than 2003 use the standardized 128-bit RSA algorithm in order to generate \glspl{otp}. Newer revisions also feature a USB port that allows the device to store custom certificates, i.e., making it a smartcard device, too. Additional form-factors, such credit card sized variant, exist, too. Each token contains a burned in seed, a random key that was generated while manufacturing the device. Since this seed needs to be known to validate the \gls{otp}, the RSA SecurID server needs to be used. The default time for the \gls{otp} amount 60 seconds, but this can be configured to, e.g., 30 seconds. The SecureID tokens are battery powered and small enough to be carried on the keyring. The SecurID can itself be protected by a \gls{pin} that is required to generate the \gls{otp}.\footcites[See][479--480]{eckert-it-sec-9}[See][296]{4351500}

Mobile applications for iOS, BlackBerry \gls{os}, BlackBerry 10, Windows Phone and Android exist, too, offering support for a soft-token based solution. Desktop applications for macOS and Windows are available, too.\footcites[See][3--6]{ibm-mfa}[See][49]{5542954}

\subsection{YubiKey}

Besides a proprietary \gls{otp} algorithm, the company Yubico is best known for their physical security tokens, the YubiKey. A variety of tokens exist, ranging from different USB-A and USB-C variants or \gls{nfc}-capable tokens to lightning connectors for the usage with iOS. Besides different connectivity, different form factors are available, too. For example Yubico offers very tiny tokens that can remain in the USB port permanently. All tokens, except the \frqq Security Key\flqq{} series, support Yubico's \gls{otp} algorithm, \gls{hotp}, \gls{totp} and \gls{u2f}, as well as static passwords and OpenPGP. The \frqq FIPS series\flqq{} are \gls{fips} 140-2 certified, i.e., their cryptographic modules are approved by the U.S. government, and usable for \gls{piv}.\footcites[See][716]{Vacca2017aa}[See][83]{Jacobs:2016:STA:2953926.2953927}[See][109]{Jacobs:2019}

\subsection{Software tokens}

While already touched briefly in the previous subsections, the software or soft token are security tokens completely available as a software, either as, e.g., a smartphone, mobile phone or desktop application. In contrast to hardware tokens, the software tokens are more easily copyable. Software tokens, especially smartphone applications, can itself protected by a password or biometric factor. Software tokens have the advantage of using device \glspl{api} such das push notifications, some of them even allow method of \frqq authentication by push notifications\flqq. Other software tokens do not generate an \gls{otp} but instead allow the user to approve or deny the authentication request.\footcites[See][717]{Vacca2017aa}[See][113]{vacca2013managing}[See][60]{Ulqinaku:2019:FPP:3317549.3323404}[See][222--223]{dasgupta2017multi}