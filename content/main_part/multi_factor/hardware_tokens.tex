\section{Security Tokens}

Besides smart cards, further \gls{mfa} solutions with possession as an additional factor are security tokens or keys. These security tokens exist as pure hardware solutions and as software-based solutions. The minimum security requirements for the cryptographic modules are defined in, e.g., the \gls{nist} \gls{fips} 140-3 standard. This section introduces the well-known security tokens \frqq RSA SecurID\flqq{} and \frqq YubiKeys\flqq. Typically, security tokens either store a private key used in public-key cryptography or the shared secret in order to generate or validate \glspl{otp}.\footcites[See][]{FIPS140-3}[See][Chapter 28.4.3]{1174011}

\subsection{RSA SecurID}

The RSA SecurID exists in several variants, both as hardware and as software tokens. First hardware revisions used a 64-bit proprietary protocol called \frqq SecurID hash function\flqq. Hardware keys newer than 2003 use the standardized 128-bit \gls{rsa} algorithm in order to generate \glspl{otp}. Newer revisions also feature a \gls{usb} port that allows the device to store custom certificates, i.e., making it a smart card device, too. Other form factors, such as credit card-sized variants, exist, too. Each token contains a burned in seed and a random key that was generated while manufacturing the device. Since this seed needs to be known to validate the \gls{otp}, the RSA SecurID server needs to be used. The default time for the \gls{otp} time-step value is 60 seconds, but this can be configured to, e.g., 30 seconds. The SecurID tokens are battery powered and small enough to be carried on the keyring. The SecurID can itself be protected by a \gls{pin} that is required to generate the \gls{otp}.\footcites[See][479--480]{eckert-it-sec-9}[See][296]{4351500}

Mobile applications for iOS, BlackBerry \gls{os}, BlackBerry 10, Windows Phone and Android exist, too, offering support for a soft-token based solution. Desktop applications for macOS and Windows are also available.\footcites[See][3--6]{ibm-mfa}[See][49]{5542954}

\subsection{YubiKey}

Besides a proprietary \gls{otp} algorithm, the company Yubico is best-known for its physical security tokens, the YubiKey. A variety of tokens exist, ranging from different \gls{usb}-A and \gls{usb}-C variants, \gls{nfc}-capable tokens to lightning connectors for the usage with iOS. Besides different connectivity, various form factors are also available. For example, Yubico offers very tiny tokens that can remain in the \gls{usb} port permanently. All tokens, except the \frqq Security Key\flqq{} series, support Yubico's \gls{otp} algorithm, \gls{hotp}, \gls{totp}, and \gls{u2f}, as well as static passwords and OpenPGP. The \frqq FIPS series\flqq{} is \gls{fips} 140-2 certified, i.e., their cryptographic modules are approved by the \gls{us} government and are usable for \gls{piv}.\footcites[See][716]{HUSEYNOV2017715}[See][83]{Jacobs:2016:STA:2953926.2953927}[See][109]{Jacobs:2019}

\subsection{Software Tokens}

As already touched briefly in the previous subsections, software tokens are security tokens completely realized as software, either as, e.g., a smartphone, mobile phone, or desktop application. In contrast to hardware tokens, the software tokens are more easily copiable. A password or biometric factor can itself protect software tokens, especially smartphone applications. Software tokens have the advantage of using device \glspl{api} such as push notifications. Some of them even allow an \frqq authentication by push notifications\flqq, where a user needs to tap on an incoming push notification to confirm the authentication. Other software tokens do not generate an \gls{otp} but instead allow the user to approve or deny the authentication request. Software tokens were available before the smartphone era, too, by using, e.g., Java MIDlets for regular mobile phones that were capable of using the \gls{wap}.\footcites[See][717]{HUSEYNOV2017715}[See][111]{ELMALIKI201475}[See][60]{Ulqinaku:2019:FPP:3317549.3323404}[See][222--223]{dasgupta2017multi}[See][3]{4300040}
