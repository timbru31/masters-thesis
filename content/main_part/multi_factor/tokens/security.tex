Smart cards and physical security tokens face common security threats. As both are authentication by possession, they are at risk of being stolen, damaged, lost, or rendered inoperable in any other way. Especially security tokens that are carried around on a keychain are exposed to the threat of being left on the desk and therefore being accessible for other people.

Further, physical tokens and smart cards are at risk of being cloned or disassembled in order to gain access to the underlying chip. Also, malicious applets for Java-enabled smart cards can try to exploit software vulnerabilities.\footcites[See][14--16]{witteman2002advances}

The first generation of RSA SecurID tokens contained a vulnerable algorithm that led to a successful adaptively chosen-plaintext attack. Even though RSA replaced the algorithm in their security tokens, in March 2011, an intruder successfully managed to gain access to RSA's internal seed and serial number database that lead to a replacement of over 40 million RSA SecurID tokens.\footcites[See][480]{eckert-it-sec-9}[See][369]{BIRYUKOV2005364}[See][8]{1324198}

Besides vulnerabilities in the company's server, both Google's Titan Key and YubiKeys suffered from vulnerable firmware. Due to the token's design, it is not possible to update the firmware, and the only security mitigation remains the replacement of affected devices.\footcites[See][]{yubikey-heise}[See][]{titan-key}

In order to attack the chips inside smart cards and security tokens, a side-channel attack, such as the \gls{dpa}, can be used. YubiKeys were successfully attacked, and their private \gls{aes} could be extracted. Therefore, an attacker was able to gain access to the Yubico \gls{otp} generation. Additionally, an attacker can target the \gls{eeprom} of a smart card, e.g., by trying to freeze and copy the values.\footcites[See][210, 212, 219]{10.1007/978-3-642-41284-4_11}[See][502--503, 509]{anderson2008security}\\
Moreover, given the fact that a smart card operates as a \gls{usb}-\gls{hid} and some RSA SecurID tokens contain a \gls{usb} port for smart card compatibility, a malware might be able to intercept the \gls{usb} communication.

As software tokens mainly generate \glspl{otp}, the threats regarding the generation by an application also apply (see \autopageref{sec:app}).
 