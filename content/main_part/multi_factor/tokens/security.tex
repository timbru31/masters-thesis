Smartcard and physical security tokens face common security threats. As both are an authentication by possession, they are at risk of being stolen, damaged, lost, or rendered inoperable in any other way. Especially security tokens that are carried around on a keychain are exposed to the threat of being left on the desk and therefore being accessible for other people.

Further, physical tokens and smartcard are at risk of being cloned or disassembled in order to gain access to the underlying chip. Also, malicious applets for Java enabled smartcards can try to exploit software vulnerabilities.\footcites[See][14--16]{witteman2002advances}

The first generation of RSA SecurID contained a vulnerable algorithm that lead to a successful adaptively chosen plaintext attack. Even though RSA replaced the algorithm in their security tokens, in March 2011 an intruder successfully managed to gain access to RSA's internal seed and serial number database that lead to a replacement of over 40 million RSA SecurID tokens.\footcites[See][480]{eckert-it-sec-9}[See][369]{BIRYUKOV2005364}[See][8]{1324198}

Besides vulnerabilities in the companies server, both Google's Titan Key and YubiKeys suffered from a vulnerable firmware. Due to token design, it is not possible to update the firmware and the only security mitigation remains the replacement of affected devices.\footcites[See][]{yubikey-heise}[See][]{titan-key}

In order to attack the chips inside smartcards and security tokens, a common attack are side-channel attacks such as a \gls{dpa}. YubiKeys were successfully attacked and their private \gls{aes} could be extracted. Therefore, an attacker was able to gain access to the Yubico \gls{otp} generation. In addition, an attacker can target the \gls{eeprom} of a smartcard, e.g., by trying to freeze and copy the values.\footcites[See][210, 212, 219]{10.1007/978-3-642-41284-4_11}[See][502--503, 509]{anderson2008security}\\
Moreover, given the fact that smartcard operate as a \gls{usb}-\gls{hid} and some RSA SecurID tokens contain a \gls{usb} port for smartcard compatibility, a malware might be able to intercept the \gls{usb} communication.

As software tokens mainly generate \glspl{otp} and the generation by an application already has been discussed, the reader is referred to \autopageref{sec:app}.
 