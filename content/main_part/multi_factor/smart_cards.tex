\section{Smartcards}

Smartcards, sometimes called chip cards or \glspl{icc}, too, are physical plastic cards, often the size of a credit card and contain an internal chip for user authentication. The chip is either exposed or can be accessed contactless. Typical examples are \gls{sim} cards, credit cards, \glspl{cac} used by the \gls{us} Department of Defense, or identity cards issued by authorities. In \gls{it}, smartcards can also store certificates and are used for computer log on. The smartcard differs from a regular storage card by having a microprocessor and an \gls{eprom} or \gls{eeprom}. It is defined in the \gls{iso} standard 7816, which also defines different sizes of smartcards. The \gls{nist} standard defines in \gls{fips} 201-2 the usage of smartcards for \gls{piv} of federal employees.\footcites[See][525--527]{eckert-it-sec-9}[See][]{iso7816}[See][6--9]{Mayes2017}[See][]{FIPS201-2}

\begin{figure}[hbt]
	\centering
	\includesvg[width=\textwidth,pretex=\relscale{0.8}]{pics/svg/smartcard_arch}
	\caption[Typical smartcard architecture]{Typical smartcard architecture\footnotemark}
	\label{fig:smartcard_arch}
\end{figure}
\footcitetexts[Source: diagram by author, based on][33]{electronic_certification_mobile_devices}[][228]{Tunstall2017}

\autoref{fig:smartcard_arch} shows the typical architecture of a smartcard chip. The \gls{rom} contains the \gls{os} of the smartcard while the \gls{ram} is used for temporary storage. The application storage uses the \gls{eeprom}. Some smartcards also contain a second processor for cryptographic operations.

Security-wise an essential requirement is that an attacker cannot access the private data on the internal chip, i.e., that the smartcard is tamper-resistant. This is, e.g., achieved by physically covering the \gls{cpu}, \gls{ram}, and \gls{eeprom} with a shield. The data stored on the smartcard can itself protected by using a \gls{pin} or biometrics, such as a fingerprint, to access the data.\footcites[See][34]{265831}[See][228]{Tunstall2017}

In the aspect of usability, the smartcard always requires dedicated hardware, either external or built-in, a smartcard reader in order to use the smartcard as an authentication method. While especially enterprise notebooks contain a smartcard slot, e.g., mobile phones do not. With the \gls{ccid} protocol which defines a \gls{usb} protocol it is at least possible to use a \gls{usb} card reader. Additionally, smartcards with an embedded \gls{jcvm} allow the execution of Java application and servlets, opening the development possibilities for smartcard application further.\footcites[See][65]{Markantonakis2017}[See][539]{eckert-it-sec-9}