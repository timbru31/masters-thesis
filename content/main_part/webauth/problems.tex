\section{Security Aspects}

The following sections build upon the security of \gls{mfa}, especially security tokens and \gls{u2f} by further concentrating on the specifics of the \wa{} and the architectural changes opposite to the \gls{u2f} protocol.

\subsection{Problems}

The problems that are transferred to the \wa{} are the ones of authentication by possession already described in \autoref{sec:possession-security} and further specified for security keys in \autoref{sec:tokens} and \gls{u2f} in \autoref{sec:u2f}. If the \wa{} is used with a physical security key, then the same threats of damage, loss, or theft exist. Besides that, if the security key itself is not protected (by, e.g., fingerprints) an attacker can easily gain access to an account if he steals or copies the authenticator. Built-in key stores in devices, such as smartphone or laptops, do not protect against theft, either. Furthermore, roaming authenticator are subject to physical attacks, in particular side-channel attacks such as a \gls{dpa}.

Security-wise the \wa{} has received little attention yet. A first security analysis showed some weaknesses. These are the following ones, described more detailed below. It has to be noted though, that these security considerations are only from one source and outlined only theoretical attack vectors. In contrast, the security of the \wa{} was formally verified and findings only resulted in privacy concern and no security issues were found.\footcites[See][]{paragon-webauth}[See][9]{FormalVerificationWebAuthn}

The first problem of the \wa{} are the registered \gls{cose} algorithms in section 11.3. A support for \gls{rsa}SSA-\gls{pkcs}\#1 v1.5 is explicitly required, making it vulnerable for the over twenty years known \frqq Bleichenbacher attack\flqq.\footcites[See][]{10.1007/BFb0055716}

Further, the \gls{ecdaa} does not specify point compression. This can lead to invalid curve attacks, where an attacker can send a chosen point that is assumed to be on the elliptic curve. However, if the point is not on the curve it can lead to the leakage of the private key. Random values for the secret key in \gls{ecdsa} expose a threat, too. It is recommended to use determinist \glspl{nonce} instead.

Additionally, the usage of the \gls{rng} is not further specified. Weak implementations might use the \textit{standard} \gls{rng} and not a suitable \gls{csprng} for the \gls{ecdaa}. Moreover, the attestation can be criticized because the specification do not require an implementer to use \gls{ecdaa} but also allow to use the attestation private key of the security token. In combination with a centralized attestation \gls{ca} this can de-crease the user's anonymity and break the principle of the un-linkability between users and \glspl{rp}.

\subsection{Mitigations}

As mitigations against theoretical security threats, the following changes should be taken into account for future revisions of the \wa{}. For example, even when the standard does not specify which \gls{rng} has to be used, an implementer of such a cryptographic \gls{api} should always know the danger of potential lack of randomness when not using a \gls{csprng}. 

The Bleichenbacher attack and padding oracle can be avoided by not implementing the RSA \gls{pkcs}\#1 v1.5 padding requirement, although this is a violation of the specification. In addition, when for instance the \gls{rp} does not implement the \gls{api} themselves, they have the control over the public key credential parameters array (\textit{pubKeyCredParams}) specified in the registration ceremony. Omitting the identifier for the RSA \gls{pkcs}\#1 v1.5 padding ensures that the key material is generated with a different algorithm.

