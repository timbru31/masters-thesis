\section{Security aspects}

\subsection{Problems}

The problems that are transferred to the \wa{} are the ones of authentication by possession already described in \autoref{subsec:possession-threat}. If the \wa{} is used with a security key, then the same risk of damage, loss or theft exist. Besides that, if the security key itself is not protected (by e.g. fingerprints) an attacker can easily gain access to an account if he steals or copies the authenticator. Built-in key stores in devices such as smartphone or laptops, do not protect against theft, either.

Security-wise the \wa{} had little attention yet. A first security analysis showed some wekanesses. These are the following ones, described more detailed below:

\begin{enumerate}
	\item Support for RSA PKCS1v1.5
	\item The usage of ECDAA
	\item Randomness of ECDAA
	\item Chosen ECDSA curves
\end{enumerate}

It has to be noted though, that these security consideations are only from one source.\footcite[See][]{paragon-webauth}