\section{Goal of the Web Authentication API}

The goal of the \wa{} is to enable \frqq the creation and use of strong, attested, scoped, public key-based credentials by web applications, for the purpose of strongly authenticating users\flqq.\footcites[See][Abstract]{w3c} Each public key credential is scoped to the \glsfirst{rp}, i.e. can not be re-used for other websites (\glspl{rp}). The authenticator has the duties and responsibilities to create, store, and access these credentials. These action always require the users consent. The user agent, i.e., browser performs the communication with the authenticators and \glspl{rp} to preserve the users privacy. Attestation ensures that each operation from an alleged authenticator is legitimate and cryptographically verifiable. The primary use cases for the \wa{} are passwordless registrations and logins, but also to provide a second-factor or to sign specific transactions.\footcites[See][Abstract, Chapter 1.2]{w3c}

\section{History and Evolution}

The \wa{} is an outcome of joint efforts between	 the \gls{fido} alliance and the \gls{w3c}. It is an outcome from preceding industry standards, namely \glsfirst{uaf} and \glsfirst{u2f}. This chapter introduces the \wa{} with focus on the technical implementations, protocols and used techniques.\footcites[See][24]{fido-ct-3}

The first specification version of the \gls{u2f} was the starting point for the development of the \wa{} in a joint efforts with the \gls{w3c}. The \gls{ctap} is based on the \gls{u2f} specification version 1.2 which complements the \wa. Both projects are part of the FIDO2 project.\footcite[See][169--170]{grimes2017hacking}
