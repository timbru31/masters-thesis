\section{One-time password}

To fully understand how the \gls{otp} works the basics and origin's have to be introduced first. In the following subsections are the required algorithms shorty described and in ... the variants of \glspl{otp}, both the \gls{hotp} and the \gls{totp} which are based on \gls{hmac}.

\subsection{MAC}

The \gls{mac} is a \textit{code}, i.e., some sort of information to protect and ensure the integrity of a \textit{message}. Integrity, besides confidentially and availably, is one of the key concepts of \gls{it} security. The \gls{mac} is built using two parameters, a secret key that both parties know and the message itself. The algorithm generates a checksum that the sender can send alongside with the message. The recipient calculates the checksum (\gls{mac}) itself from the retrieved message. If it differs then the message has been manipulated or there might have been a faulty transmission. Technically the \gls{mac} can be generated with, e.g., cryptographic-hash functions (\gls{hmac}) or using block ciphers such as \gls{cbc-mac}.\footcites[See][565]{320284}[See][163--168]{anderson2008security}


The \gls{mac} is standardized in different standards from various instituions, such as \gls{nist} \gls{fips} 198-1, the \gls{bsi} technical guideline TR-02102-1 (\frqq Cryptographic Mechanisms: Recommendations and Key Length\flqq{}) or the \gls{iso} norm ISO/IEC 9797-1 and ISO/IEC 9797-2.\footcites[See][]{FIPS198}[See][]{bsi2019recommendations}[See][]{iso9797-1}[See][]{iso9797-2}

Sometimes the \gls{mac} is also called \gls{mic} in order to avoid confusion with the \gls{macaddress} address used in network protocols. Further, the \gls{mic} would not prove authenticity since an attacker can just modify the message and re-generate the \gls{mic} of the modified message.\footcites[See][60-62]{265831}

\subsection{HMAC}

\gls{hmac} Code is an extension of a \gls{mac} and standardized in \gls{rfc} and \gls{nist} abc.
\cite{krawczyk1997rfc}

\subsubsection{HTOP}
\label{subsubsec:hotp}

HMAC-based One-time Password algorithm, counter based. \gls{rfc} 899. Configurable length (6-10). Default SHA1. Truncation of HMAC
\cite{m2005rfc}

\subsubsection{TOTP}
\label{subsubsec:totp}

Time based instead of counter based. \gls{rfc} 123 and \gls{oath}.
\cite{m2011rfc}

\begin{figure}[hbt]
	\centering
	\includesvg[width=\textwidth,pretex=\relscale{1}]{pics/svg/2fa_flow}
	\caption[Exemplary \gls{mfa} authentication flow]{Exemplary \gls{mfa} authentication flow\footnotemark}
	\label{fig:2fa_flow}
\end{figure}
\footnotetext{Source: diagram by author}

\autoref{fig:2fa_flow} shows an example authentoiation

\subsubsection{pros}

\begin{enumerate}
	\item Collisions in MD5 or SHA1 are no problem, already stated/analyzed in the RFC
\end{enumerate}

\subsubsection{cons}

"Just an algorithm"

\begin{enumerate}
	\item synchronization
	\item invalidation
	\item nobody knows how the algorithm is implemented (RFC = no standard)
	\item Differences (e.g. Steam - only 5 digits, limited Alphabet)
	\item Brute Force if server does not limit
	\item Not phishing resistant
\end{enumerate}