\chapter{Single-factor authentication}

% TODO Challenge-Response, Zero Knowledge, Salt, NONCE

\section{Threats}
\label{sec:one-factor-threats}

\subsection{Knowledge}

\frqq Passwords are both the bane and the foundation of [...] security.\flqq{}\footcite[206]{517355}, yet the most used authentication method remains knowledge. While it seems the simplest method to use, it comes with a lot of downsides, too, which are explained in detail in this subsection. Because the knowledge should be remembered by the user, the human brain has difficulties to remember different a unique, secure password, \gls{pin} or secret questions for every unique account the user has registered for. The average amount of different accounts a user has is 10, not including, e.g., credit card \glspl{pin}. Because of this fact, the user does a couple of insecure things:

\begin{enumerate}[label=(\alph*)]
	\item using the same secret knowledge for multiple account
	\item writing down the username and passwords, e.g., on a piece of paper that's accessible easily for others, storing \glspl{pin}, e.g., in the briefcase or saving an unencrypted file on their computer or smartphone
\end{enumerate}

This enables an attacker to easily steal the login credentials of a user. Written down post it notes enable any pyhsical attacker to steal the credentials, it might be captured by a camera, too. Leaving an unencrypted file on the computer enables computer viruses and trojans to send the file to an attacker, mobile devices are affected, too. Writing down the banking \gls{pin} and storing it in the same briefcase as the credit card even annuls the \gls{2fa} example of possession and knowledge.

Regarding the security of security questions, it has to be noted that these might even decrease the security. Common examples of questions such as \frqq middle name, birth name of a parent, name of a pet, ...\flqq{} can be answered by relatives and friends, enabling a a malicious insider attack or some might be answerable by employing a social engineering attack.

Further, especially true for passwords, it's not known to the user what the service provider does in order to protect the security of the passwords. As security breaches happen nearly on a daily basis, it's important to protect the password of the user. For instance, if the passwords are stored in a database, they can

\begin{enumerate}[label=(\alph*)]
	\item unencrypted (worst case)
	\item hashed, but not salted (base case)
	\item hashed and salted (best case)
\end{enumerate}

It is pretty obvious that unencrypted passwords in a database render the biggest threat, especially when re-used. Along with the e-mail or username an attacker can probably use the stolen credentials for other accounts, too, or in case of an e-mail provider breach, just re-issue a new password with the \flqq forgotten password \frqq{} mechanism. Even if the password is hashed, but not salted, it renders the credentials at risk. Weaker hashing algorithms such as MD5 or SHA1 might be broken in the futurue, but besides that if it's a weak password, too, the hash might already been reversed. Having the hashed password lsit, enables the attacker to execute a brute force attack in order to reverse as many hashes, as possible or just searching it in databses that contain billion of reversed hash values.

The best protection is a uneique salt for each password, wich decreases the risk of a successful rainbow or brute force attack dramatically. In this scenario each password is not only hashed, but salted, too.

\begin{enumerate}
	\item secret might be known by others, too (e.g. security questions)
	\item guessing
	\item brute force
\end{enumerate}

\begin{figure}
	\begin{bchart}[min=5, max=55, step=10, unit=\%, width=.65\textwidth]
		\bcbar[label=All passwords, color=red]{7}
		\medskip
		\bcbar[label=Most of the passwords, color=red]{20}
		\medskip
		\bcbar[label=Some passwords, color=red]{45}
		\bigskip
		\bcbar[label=Don't know, color=yellow]{7}
		\bigskip
		\bcbar[label=None of the passwords, color=green!50]{20}
	\end{bchart}
	\caption{Percentage of password re-usage in the United States in 2018}
	\label{fig:us-password-reuse}
\end{figure}

	
\subsection{Possession}
\label{subsec:possession-threat}

The primary risks of authentication by possession are that it is not tied to the user itself and can be lost or even worse stolen by an attacker. Besides, that possession factors can be shared between multiple users, allowing attacks such as a malicious insider attack. Often the possession factors are not protected itself so, e.g., a keycard to open a door can be used by the attacker, too.

Another usage implication is that it has to be carried with the user and can be forgotten which makes the authentication impossible if no access to the possession is possible and no backup or different authentication methods are available. Another risk is that possession can be damaged or destroyed. For example, carrying security keys on a keyring exposes them to damage by a fall or liquids.\footcites[See][263--264]{shostack2014threat}

Especially possessions that use wireless transmissions such as \gls{ble}, \gls{nfc}, or \gls{rfid} can be copied even over some distances. For instance, an attacker could copy credit cards in crowded places such as trains or buses.\footcite{6892730}

\subsection{Biometrics}

In contrast to possession and knowledge, the biometric trait cannot easily be stolen. While it can be copied, e.g. the fingerprint from high-resolution photographs or 
face models to circumvent face recognition systems.\footcites[][]{185181}[][]{220566} In the recent past, researchers could copy both German Chancellor's Angela Merkel's iris and the fingerprint of Ursula von der Leyen, the now the elected President of the European Commission, from high-resolution photographs.\footcite{ccc-merkel} It has to be taken into account though, that especially the so-called latent fingerprints are nearly left everywhere, i.e., the security of biometrics heavily relies on the chosen biometric trait.\footcite[See][299]{Vacca2017aa}

Further implications are that the biometric characteristics can change over time or be temporarily unavailable because of injuries. While some can heal over time, others, especially scars, can permanently change the biometric trait and therefore render it unusable. Also, each time the user authenticates with biometrics, a new sample of the trait is gathered and compared to the stored one. Because the recent probe will never be 100\% identical compared to the stored one (\frqq intra-user variants\flqq), a threshold needs to be defined, which allows or denies the authentication attempt. Setting the threshold to a too low value increases the risk of the \gls{far}, while a too high value decreases the usability and increases the \gls{frr}.\footcite[See][13--17, 52]{Jain2011}\\Traits such as facial recognition must also be usable with different amounts of facial hair, hairstyles, with and without glasses.\footcite[See][98]{Jain2011}

Another high risk is data privacy and security. Over 50\% of the users fear about data usage, both legitimate and abusive, and collection of their biometrics, yet the majority of the user states that biometrics is the most secure authentication compared to, e.g., passwords and \glspl{pin}.\footcite[See][8]{ibm-security} It is crucial that the stored biometric probe is not accessible by third parties nor shared with them. For example, a theft of smartphones should not mean theft of the biometrics, e.g., fingerprint, too.

However the major threat remains the difficulty to replace a compromised biometric template. While, e.g., a password or a security key can be changed or replaced, for instance a fingerprint can't be altered, changed or replaced since it remains the same for the whole lifespan of a person. To counter this threat it is advised to use, for instance, only a hash of the fingerprint and not store the \textit{image} of the fingerprint itself.\footcite[See][266]{shostack2014threat}

Further, it is necessary to respect the quality and availability of the sensor. If a sensor is damaged, too cheap or the surface is, for example, dirty, then the authentication and especially the usability suffers.\footcite[See][37]{265831}

\subsection{Further methods}

A high risk of location-based authentication is the spoofing of the actual location by an attacker. An attacker can choose different attack vectors, such as spoofing the source IP address that tries to access a system. Another form of spoofing is the \gls{gps} spoofing where an attacker modifies the actual \gls{gps} by broadcasting false information. Further, the Caller ID spoofing technique can be used with VoIP to disguise the location. Besides these techniques, the most common variant remains the usage of a \gls{vpn} network or \gls{dns} proxy to hide the genuine location.\footcites[See][138--145]{517355}[See][Chapter 4.5.3]{30270}[See][115--116,133]{eckert-it-sec-9}

For time-based authentication, an attacker could use attacks against the \gls{ntp} in order to either gain access of the verification system or to modify the synchronized time in order to allow the login attempt to succeed.\footcite[See][]{malhotraattacking}

\subsection{General threats}

% TODO writeme
General threat: security of transmission!

\subsubsection{Initialization/Registration/Enrollment}

A more general threat is the registration or initialization of the authentication. The user has to make sure that no attacker can intercept or copy the required enrollment data. For instance, if a malware comprises a user’s computer and installs a keylogger, then an entered password is no longer a secret and is compromised. A computer virus could also intercept a USB connection from a security key, both when registering the device and while using it. Furthermore, the user needs to make sure that his enrollment process is not observed from, e.g., a surveillance camera, a hacked webcam, or a colleague from behind.\footcites[See][152--153]{10.1007/978-3-642-39235-1_9}[See][61]{Ulqinaku:2019:FPP:3317549.3323404}

\subsubsection{Transmission}

% TODO writeme