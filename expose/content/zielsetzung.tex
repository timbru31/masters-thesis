\chapter{Zielsetzung und Abgrenzung}

\section{Zielsetzung}

Zielsetzung dieser Master Thesis ist eine Einführung in die Multi-Faktor-Authentifizierung und verschiedene gängige Faktoren (Besitz, Wissen, Merkmal) inklusive der technischen Funktionsweise, Nutzungsmöglichkeiten im Web und potenziellen Sicherheitsproblemen. Die Web Authentication API soll als Alternative bzw. mögliche Ergänzung hierzu vorgestellt werden. Dabei muss die Frage geklärt werden, inwiefern diese die Sicherheit und den Nutzerkomfort erhöhen kann. Dabei spielt die Bewertung der Sicherheit der Web Authentication API eine entscheidene Rolle.

\section{Zielgruppe}

Die Zielgruppe dieser Master Thesis sind technisch versierte Leser, die ein Verständnis für Datenschutz und -sicherheit besitzen und die (mathematische) Funktionsweise hinter 
Algorithmen wie RSA, elliptischen Kurven und symmetrischen Schlüsselaustausch kennen. Des Weiteren richtet sich diese Master Thesis an interessierte \mbox{(Web-) Entwickler}, die den Einsatz alternativer oder ergänzender Registrierungsverfahren und Multi-Faktor-Authentifizierungen durch den Einsatz von asymmetrisches Kryptosystemen verstehen und ihre Vor- und Nachteile nachvollziehen wollen.

\section{Abgrenzung}

Bestehende Algorithmen und Konzepte werden, soweit nicht für den Verlauf der Arbeit benötigt, nicht näher erläutert. Außerdem soll diese Master-Thesis keine reine Kryptoanalyse werden, sondern auch auf andere Aspekte wie die Benutzbarkeit durch die Benutzer, technische Umsetzbarkeit und Unterstützung eingegangen werden, um zu evaluieren inwiefern die Web Authentication API genutzt werden kann. Die Bereiche OAuth 2.0 bzw. OpenID Connect sowie Single Sign On (SSO) sind ebenfalls nicht Fokus dieser Arbeit.
