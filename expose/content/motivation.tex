\chapter{Problemstellung und Motivation}

Der sichere Umgang mit Passwörtern stellt für viele Anwender noch ein Problem dar.\footnote{Nier, H. (2017, Januar 23). Der große Passwort-Stress [Digitales Bild]. Zugriff am Januar 23, 2019, von https://de.statista.com/infografik/7705/der-grosse-passwort-stress/.} Häufig werden Passwörter von verschiedenen Seiten wiederverwendet, teilweise auch die privaten Passwörter mit denen aus dem beruflichen Umfeld vermischt. Dies erzeugt ein hohes Risiko für die (vertraulichen) Daten der Nutzer. Hinzu kommt, dass weniger als die Hälfte der Deutschen im Jahr 2019 Multi-Faktor-Authentifizierung nutzen und nur 10\% der Deutschen in 2018 einen Passwortmanager genutzt haben. Aktuell merken sich über die 70\% der Befragten die Passwörter noch im Kopf oder schreiben sie in Klartext auf einen Zettel.\footnote{ARD, \& Infratest dimap. (n.d.). Was tun sie, um sich vor einem Missbrauch Ihrer persönlichen Daten zu schützen?. In Statista - Das Statistik-Portal. Zugriff am 23. Januar 2019, von https://de.statista.com/statistik/daten/studie/955802/umfrage/massnahmen-von-internetnutzern-zum-schutz-vor-datenmissbrauch-in-deutschland/.}\\
Zeitgleich steigt die Anzahl an erfassten Fällen von Cyberkriminalität weiter an\footnote{Bundeskriminalamt. (n.d.). Polizeilich erfasste Fälle von Cyberkriminalität im engeren Sinne* in Deutschland von 2004 bis 2017. In Statista - Das Statistik-Portal. Zugriff am 23. Januar 2019, von https://de.statista.com/statistik/daten/studie/295265/umfrage/polizeilich-erfasste-faelle-von-cyberkriminalitaet-im-engeren-sinne-in-deutschland/.
} und auch zum Beispiel Phishing bleibt ein großes Risiko. Zwar können Multi-Faktor-Authentifizierungen gegen Bedrohungen wie Brute-Force oder geklaute Passwörter Abhilfe schaffen, aber gegen gut getarnte Phishing-Attacken sind diese nutzlos. Des Weiteren gilt beispielsweise der SMS Verkehr nicht mehr als sicher, trotz dessen werden viele zweite Faktoren über diesen unsicheren Weg übermittelt.
Diesen negativen Trends steht jedoch eine neue Programmierschnittstelle gegenüber -- die Web Authentication API. Diese befindet sich in Entwicklung für die weit verbreiteten Browser wie Chrome, Edge, Safari und Firefox. Sie ermöglicht eine sichere Registrierung, Login und Zwei-Faktor-Authentifizierung ganz ohne das Generieren, Merken oder Aufbewahren von Passwörtern durch die Nutzung von asymmetrischen Kryptosystemen. Die privaten Schlüssel können sowohl auf externen Geräten wie USB Sticks gespeichert werden, als auch durch eingebaute Sensoren in den Geräten wie Fingerabdruckscanner gespeichert und geschützt werden.