%% allgemeine Textformatierung

\documentclass[
	oneside,			%				einseitig
	a4paper,			%				DIN A4-Format
	12pt					%				Schriftgröße
]{scrreprt}

% Für die Verzeichnisse für Abkürzungen, Fremdwörter, Symbole
\usepackage[]{artlos,artlob,artlow}

%% deutsche Sprachanpassungen (Silbentrennung, Umlaute usw.)
\usepackage{ngerman}

%% keine Ahnung, is aber immer drin :-)
\usepackage[T1]{fontenc}
\usepackage[utf8]{inputenc}

%% nicht bitmap, sondern postscript (type 1, vector)
\usepackage[]{pslatex}

%% Package zum einfärben von Text und anderen Objekte
%\usepackage{color}

%% Package zum einbinden von Grafiken
\usepackage[]{graphicx}

%% Package für Index (Register)
\usepackage{makeidx}
\makeindex
\renewcommand{\indexname}{Stichwortverzeichnis}

%% Package für Kopfzeilen
\usepackage{fancyhdr}
% Kopfzeilenformat festlegen
\rhead{}
\headheight = 14.5pt

%% Package zum einbetten von Verweisen in die Ausgabe (DVI, PDF)
\usepackage[
	bookmarksopen=true,			% öffnet alle Lesezeichen (nur in PDF)
	plainpages=false				% ermöglicht mehrere Seitenzahlformate
]{hyperref}

%% Package für Abkürzungsverzeichnis (Glossar)
\usepackage{nomencl}
	\let\abbrev\nomenclature
	\renewcommand{\nomname}{Abkürzungsverzeichnis}
	\setlength{\nomlabelwidth}{.25\hsize}
%	\renewcommand{\nomlabel}[1]{#1 \dotfill}
	\setlength{\nomitemsep}{-\parsep}
% Paketabhaengig heissen die Befehle makenomenclature / printnomenclature
%                 oder makeglossary / printglossary !!!!
	% \makenomenclature
	\makeglossary
%	\let\listofabbreviations\printnomenclature  % Paketabhaengig !!!!
	\let\listofabbreviations\printglossary

%% Hervorhebung für das Abkürzungsverzeichnis
\usepackage[normalem]{ulem}
	\newcommand{\markup}[1]{\underline{#1}}

% Seitenformat festlegen
\textwidth14.5cm
\textheight24.5cm
\topmargin-15mm

% 1.5 zeilig
\linespread{1.25}\selectfont

%% Package zum formatieren von Tabellen
\usepackage{array}

% kein Einrücken am Absatzanfang
\parindent 0pt 

% variabler Abstand zwischen Wörtern, um Blocksatz zu wahren und weniger zu trennen
\sloppy

\usepackage{xcolor}
