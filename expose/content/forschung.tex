\chapter{Stand der Forschung}

Die Themenbereiche wie Cyberkriminalität, Multi-Faktor-Authentifizierung und sichere Passwörter sind ausreichend erforscht, ebenso die dahinter zum Einsatz kommenden Algorithmen und mathematischen Grundlagen. Dazu zählen auch bekannte Schwachstellen und etwaige Lösungen für diese. Vorarbeiten im Bereich Universal Two Factor (U2F) existieren ebenfalls, jedoch ohne den Bezug zur Web Authentication API und ihrem Einsatz.\footnote{Korkmaz, M. (2017). Grundlagen der FIDO-Authentifizierung und Vergleich mit traditionellen Authentifizierungsverfahren (Doctoral dissertation, Hochschule für Angewandte Wissenschaften Hamburg). Zugriff am 23. Januar 2019, von http://edoc.sub.uni-hamburg.de/haw/volltexte/2017/4020/pdf/Bachelorarbeit\_MuratKorkmaz.pdf}\footnote{Angelogianni, A. (2018). Analysis and implementation of the FIDO protocol in a trusted environment (Master's thesis, University of Piraeus). Zugriff am 23. Januar 2019, von http://dione.lib.unipi.gr/xmlui/handle/unipi/11387}\\

Da sich Web Authentication API aktuell noch in der Entwicklung befindet, ist der Stand der Forschung im Vergleich zu den bekannten Multi-Faktor-Authentifizierungsverfahren weniger fortgeschritten. Die aus dem FIDO Standard resultierende Web Authentication API nutzt zwar bekannte mathematische Grundlagen, die technische Umsetzung und Einbindung in produktive Systeme ist aber weitestgehend noch nicht erfolgt. Ebenso gibt es bisher kaum Analysen zu ihrer Sicherheit und den möglichen Schwächen.