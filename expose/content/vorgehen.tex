\chapter{Vorgehen}


Zunächst soll in dem ersten Abschnitt dieser Master Thesis der Leser in Grundlagen von Passwörtern und Authentifizierung eingeführt werden. Hierfür werden die verschiedenen Bereiche
\begin{enumerate}
	\item Passwörter ohne zweiten Faktor (``One-Faktor-Authentication``)
	\item Multi-Faktor-Authentifizierung
	\item Web Authentication API
\end{enumerate}

kurz vorgestellt und voneinander abgegrenzt.\\

Anschließend wird auf die Probleme und Risiken von ``One-Faktor-Authentication`` eingegangen. Im dritten Abschnitt werden die verschiedenen Faktoren (beispielsweise zeitbasierte Faktoren, SMS oder biometrische Merkmale) näher analysiert und ihre technische Funktionsweise beschrieben. Außerdem werden die Faktoren jeweils auf ihre Sicherheit und Schutz gegen mögliche Angriffe wie Phishing oder Man in the middle (MITM) untersucht.\\
Darauf aufbauend wird die Web Authentication API im vierten Abschnitt detailliert beschrieben. Hierbei wird im Detail auf die technische Funktionsweise eingegangen und vor welchen Angriffsmöglichkeiten der Einsatz schützen kann, jedoch auch welche etwaigen Sicherheitsrisiken vorhanden sind. Hierfür wird auf einen technischen Proof of Concept zurückgegriffen und die Web Authentication beispielhaft implementiert.\\

Im fünften Abschnitt wird die Web Authentication API mit der Multi-Faktor-Authentifizierung verglichen. Dabei wird untersucht, inwiefern sich die Web Authentication API als Ersatz für Multi-Faktor-Authentifizierung eignet und/oder ob diese komplementär eingesetzt werden können.

Abschließend erfolgt eine Bewertung anhand der gewonnenen Erkenntnisse aus dem vorherigen Abschnitt sowie ein Fazit mit Ausblick.

